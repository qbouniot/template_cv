%%%%%%%%%%%%%%%%%%%%%%%%%%%%%%%%%%%%%%%%%
% Medium Length Professional CV
% LaTeX Template
% Version 2.0 (8/5/13)
%
% This template has been downloaded from:
% http://www.LaTeXTemplates.com
%
% Original author:
% Trey Hunner (http://www.treyhunner.com/)
%
% Important note:
% This template requires the resume.cls file to be in the same directory as the
% .tex file. The resume.cls file provides the resume style used for structuring the
% document.
%
%%%%%%%%%%%%%%%%%%%%%%%%%%%%%%%%%%%%%%%%%

%----------------------------------------------------------------------------------------
%	PACKAGES AND OTHER DOCUMENT CONFIGURATIONS
%----------------------------------------------------------------------------------------

\documentclass{resume} % Use the custom resume.cls style

\usepackage[utf8]{inputenc}
\usepackage[french]{babel}
\usepackage[T1]{fontenc}
\usepackage[left=0.75in,top=0.6in,right=0.75in,bottom=0.6in]{geometry} % Document margins
\usepackage{enumitem}
\usepackage{hyperref}


\name{Quentin Bouniot} % Your name
\addressone{6 rue Brézin \\ Paris, France 75014} % Your address
\addresstwo{(+33)~$\cdot$~626072685 \\ quentin.bouniot@gmail.com} % Your phone number and email
\addressthree{\href{https://qbouniot.github.io}{Personal page} \\ \href{https://github.com/qbouniot}{Github} \\ \href{https://www.linkedin.com/in/quentin-bouniot/}{LinkedIn}} % Your secondary addess (optional)


\begin{document}

%----------------------------------------------------------------------------------------
%	EDUCATION SECTION
%----------------------------------------------------------------------------------------

\begin{rSection}{Education}

\begin{rSubsection}{CEA-List, Université Paris-Saclay / Université Jean-Monnet, Saint-Étienne}{2019-Present}{PhD in Computer Science}{Paris, France}
    \item[] Thesis: "Few-Shot Learning: Application to Object Detection \\ and Semantic Instance Segmentation",\\ Advisor: Prof. Amaury Habrard
    
\end{rSubsection}

% {\bf PhD in Computer Science} \hfill {\em 2019-Today} \\
% \emph{Few-Shot Learning: Application to Object Detection and Semantic Instance Segmentation} \\
% CEA List - Université Jean-Monnet-Saint-Étienne \\
% Advised by Prof. Amaury Habrard \\

\begin{rSubsection}{CentraleSupélec, Université Paris-Saclay}{2015-2019}{Engineering degree - M.Sc in Computer Science}{Paris, France}
    \item[] Majoring in Robotics and AI
    
\end{rSubsection}

% {\bf CentraleSupélec, Paris, France} \hfill {\em 2015-2019} \\
% Master's Degree \\
% Majoring in Robotics and AI \smallskip \\

\begin{rSubsection}{Université de Lorraine}{2018-2019}{M.Sc in Computer Science}{Nancy, France}
    \item[] Majoring in Machine Learning, Vision and Robotics, \\
            with High Honors - rank 1/18
    
\end{rSubsection}

% {\bf Université de Lorraine, Nancy, France} \hfill {\em 2018-2019} \\
% Master's Degree \\
% Majoring in Machine Learning, Vision and Robotics \smallskip \\

\begin{rSubsection}{Higher School Preparatory Classes}{2013-2015}{Section Mathematics and Physics}{Bordeaux, France}
    \item[] Intensive preparatory course for national competitive exams to the \emph{grandes écoles}.
\end{rSubsection}
% {\bf Preparatory classes, Bordeaux, France} \hfill {\em 2013-2015} \\
% Section Mathematics and Physics \smallskip \\

\begin{rSubsection}{Baccalauréat}{2013}{Major in Mathematics}{Tahiti, French Polynesia}
    \item[] with Highest Honors (\emph{Mention Très bien})
    
\end{rSubsection}
% {\bf Baccalauréat, Tahiti, French Polynesia} \hfill {\em 2013} \\
% Major in Mathematics \\
% High Honors \smallskip \\

\end{rSection}

%----------------------------------------------------------------------------------------
%	WORK EXPERIENCE SECTION
%----------------------------------------------------------------------------------------

\begin{rSection}{Work Experience}

\begin{rSubsection}{CEA-List, Université Paris-Saclay}{April - September 2019}{AI Research Intern}{Paris, France}

\item[] Studying the impact of adversarial examples on person re-identification systems.
\item[] Improving the robustness of person re-identification systems using deep learning.
\end{rSubsection}

%------------------------------------------------

\begin{rSubsection}{SmartBuild Asia}{February - August 2018}{Data Scientist Intern}{Kuala Lumpur, Malaysia}
\item[] Automatic gathering of online complex data and matching with existing information.
\item[] Matching millions of unstructured sentences in both Malay and English to the construction project they were referring to.
\item[] Summarization of web pages in a few sentences with high accuracy.

\end{rSubsection}

%------------------------------------------------

\begin{rSubsection}{Orange France}{July 2017 - January 2018}{Intern in the Department of Cognitive Computer Science}{Paris, France}
\item[] Development of tools for Conversational Agents.
\item[] Architecture development for conversational agents that support collaborative learning.
\end{rSubsection}

\end{rSection}

\pagebreak

\begin{rSection}{Teaching}

\begin{rSubsection}{Algorithms and complexity}{2020-2022}{Course Lecturer at CentraleSupélec, Université Paris-Saclay}{Paris, France}
    \item[] First year Computer Science course for the main engineering track at CentraleSupélec.
\end{rSubsection}

\begin{rSubsection}{Factory-AI for Deep Learning Purposes}{2022}{Tutorial at CEA-List, Université Paris-Saclay}{Paris, France}
    \item[] Tutorial for several CEA-List laboratories on the use of the internal HPC cluster based on Slurm for deep learning experiments.
\end{rSubsection}

\end{rSection}

%----------------------------------------------------------------------------------------
%	PUBLICATIONS SECTION
%----------------------------------------------------------------------------------------

    \begin{rSection}{Publications}

    \textsc{Patents}

    \begin{itemize}[label=$\cdot$]
        \item \textbf{Bouniot, Quentin}, Romaric Audigier, Angélique Loesch. (2020) \emph{Méthode d'apprentissage d'un réseau de neurones pour le rendre robuste aux attaques par exemples contradictoires} (French Patent No. FR3116929A1). Institut national de la propriété industrielle (INPI). \\
        \href{https://data.inpi.fr/brevets/FR3116929}{Patent link}

        \item \textbf{Bouniot, Quentin}, Romaric Audigier, Angélique Loesch. (2020) \emph{Learning method for a neural network for rendering it robust against attacks by contradictory examples} (European Patent No. EP4006786A1). European Patent Office (EPO). \\ \href{https://worldwide.espacenet.com/publicationDetails/description?DB=&ND=3&bcId=0&locale=fr_EP&return=true&FT=D&date=20220601&CC=EP&NR=4006786A1&KC=A1#}{Patent link}
    \end{itemize}

    \textsc{International Conference Proceedings}

    \begin{itemize}[label=$\cdot$]
        \item \textbf{Bouniot, Quentin}, et al. "Proposal-Contrastive Pretraining for Object Detection from Fewer Data." \emph{International Conference on Learning Representations}. 2023. \\ (\textbf{Notable top 25\%}) \href{https://openreview.net/forum?id=gm0VZ-h-hPy}{Paper link}

        \item \textbf{Bouniot, Quentin}, et al. "Towards Few-Annotation Learning for Object Detection: Are Transformer-Based Models More Efficient?." \emph{Proceedings of the IEEE/CVF Winter Conference on Applications of Computer Vision.} 2023. \\ \href{https://openaccess.thecvf.com/content/WACV2023/html/Bouniot_Towards_Few-Annotation_Learning_for_Object_Detection_Are_Transformer-Based_Models_More_WACV_2023_paper.html}{Paper link}

        \item \textbf{Bouniot, Quentin}, et al. "Improving Few-Shot Learning Through Multi-task Representation Learning Theory." \emph{Computer Vision–ECCV 2022: 17th European Conference, Tel Aviv, Israel, October 23–27, 2022, Proceedings}, Part XX. Cham: Springer Nature Switzerland, 2022. \\ \href{https://arxiv.org/abs/2010.01992}{Paper link} \href{https://github.com/CEA-LIST/MetaMTReg}{Github link}

        \item \textbf{Bouniot} \& Redko, "Understanding Few-Shot Multi-Task Representation Learning Theory", \emph{ICLR Blog Track}, 2022. \\ \href{https://iclr-blog-track.github.io/2022/03/25/understanding_mtr_meta/}{Blog post link}

        \item \textbf{Bouniot, Quentin}, Romaric Audigier, and Angelique Loesch. "Optimal transport as a defense against adversarial attacks." \emph{2020 25th International Conference on Pattern Recognition (ICPR)}. IEEE, 2021. \\ \href{https://arxiv.org/abs/2102.03156}{Paper link} \href{https://github.com/CEA-LIST/adv-sat}{Github link}

        \item \textbf{Bouniot, Quentin}, Romaric Audigier, and Angelique Loesch. "Vulnerability of person re-identification models to metric adversarial attacks." \emph{Proceedings of the IEEE/CVF Conference on Computer Vision and Pattern Recognition Workshops.} 2020. \\ (\textbf{DeepMind Travel Award.}) \href{https://openaccess.thecvf.com/content_CVPRW_2020/html/w47/Bouniot_Vulnerability_of_Person_Re-Identification_Models_to_Metric_Adversarial_Attacks_CVPRW_2020_paper.html}{Paper link} \href{https://github.com/CEA-LIST/adv-reid}{Github link}
    \end{itemize}

    \textsc{Workshops}

    \begin{itemize}[label=$\cdot$]
        \item \textbf{Bouniot, Quentin}, et al. "Putting Theory to Work: From Learning Bounds to Meta-Learning Algorithms." \emph{4th Workshop on Meta-Learning (MetaLearn) at NeurIPS 2020}.
    \end{itemize}

    \textsc{National Conferences}
    \begin{itemize}[label=$\cdot$]
        \item \textbf{Bouniot, Quentin}, et al. "Vers une meilleure compréhension des méthodes de méta-apprentissage à travers la théorie de l'apprentissage de représentations multi-tâches." \emph{Conférence sur l'Apprentissage Automatique (CAp)}. 2021. \\ (\textbf{Oral Presentation})
    \end{itemize}

\end{rSection}

\begin{rSection}{Professional Services}

\textsc{Peer Review}
\begin{itemize}[label=$\cdot$]
    \item International Conference on Machine Learning (ICML), 2021-2023
    \item IEEE/CVF Winter Conference on Applications of Computer Vision (WACV), 2023
    \item NeurIPS Workshop on Meta-Learning (MetaLearn), 2020-2022
    \item ICML Workshop on Pre-training: Perspectives, Pitfalls, and Paths Forward, 2022
    \item International Conference on Automated Machine Learning (AutoML), 2022
    \item International Conference on Learning Representations (ICLR), 2022
    \item IEEE Transactions on Pattern Analysis and Machine Intelligence (TPAMI), \\ Special Issue on \emph{Learning with Fewer Labels in Computer Vision}, 2021
    \item Neural Information Processing Systems (NeurIPS), 2021
\end{itemize}
    
\end{rSection}
    
    %----------------------------------------------------------------------------------------
    %	TALKS SECTION
    %----------------------------------------------------------------------------------------
    
    \begin{rSection}{Invited Talks}
    
    \begin{rSubsection}{DataIA Workshop "Safety \& AI"}{23th September 2020}{CentraleSupélec, Université Paris-Saclay}{Paris, France}
    \item[] Presentation of our published work on adversarial attacks against Person Re-Identification systems and their defenses.
    \end{rSubsection}

    \begin{rSubsection}{GdR ISIS - "Towards pragmatic learning \\ in a context of limited labeled visual data"}{26th November 2021}{Paris Nord}{Paris, France}
    \item[] Presentation of our work on improving Few-Shot Learning through Multi-task Representation Learning Theory.
    \end{rSubsection}
    
    \end{rSection}

%----------------------------------------------------------------------------------------
%	TECHNICAL STRENGTHS SECTION
%----------------------------------------------------------------------------------------

\begin{rSection}{Technical Skills}

    \begin{tabular}{ @{} >{\bfseries}l @{\hspace{6ex}} l }
    Computer Languages & Python, C++, Matlab, Java, Javascript \\
    Machine Learning & Pytorch, Keras, Tensorflow, Scikit-Learn \\
    HPC & Slurm \\
    Systems & Unix, Windows, SQL, Git
    \end{tabular}
    
    \end{rSection}
    
    %----------------------------------------------------------------------------------------
    %	LANGUAGES SECTION
    %----------------------------------------------------------------------------------------
    
    \begin{rSection}{Languages}
    
    \begin{tabular}{ @{} >{\bfseries}l @{\hspace{6ex}} l }
    French & Native Language \\
    English & C1 - CentraleSupélec TOEFL ITP: 610/677 \\
    Japanese & A2
    \end{tabular}
    
    \end{rSection}

%----------------------------------------------------------------------------------------
%	VOLUNTARY SERVICE SECTION
%----------------------------------------------------------------------------------------

% \begin{rSection}{Voluntary Services}

% \begin{rSubsection}{Supélec Rézo}{2015 - 2017}{Executive Board member}{Paris, France}
% \item[] Management and maintenance of the IT and network infrastructure of the CentraleSupélec campus (about 1000 users).
% \item[] Administrator of several servers.
% % \item[] \vfill
% \end{rSubsection}

% \begin{rSubsection}{CentraleSupélec Gaming club}{2015 - 2017}{President}{Paris, France}
% \item[] Manage, organize and promote video-games events on the CentraleSupélec campus.
% \end{rSubsection}

% \begin{rSubsection}{Student Gaming Network}{2015 - 2017}{Member}{Paris, France}
% \item[] Federation of French students' associations of eSport.
% \item[] Organize every year the \emph{Student Gaming League}, an eSport tournament for students in France.
% \end{rSubsection}
% \end{rSection}


%----------------------------------------------------------------------------------------

\end{document}
