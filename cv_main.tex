%%%%%%%%%%%%%%%%%%%%%%%%%%%%%%%%%%%%%%%%%
% Medium Length Professional CV
% LaTeX Template
% Version 2.0 (8/5/13)
%
% This template has been downloaded from:
% http://www.LaTeXTemplates.com
%
% Original author:
% Trey Hunner (http://www.treyhunner.com/)
%
% Important note:
% This template requires the resume.cls file to be in the same directory as the
% .tex file. The resume.cls file provides the resume style used for structuring the
% document.
%
%%%%%%%%%%%%%%%%%%%%%%%%%%%%%%%%%%%%%%%%%

%----------------------------------------------------------------------------------------
%	PACKAGES AND OTHER DOCUMENT CONFIGURATIONS
%----------------------------------------------------------------------------------------

\documentclass{resume} % Use the custom resume.cls style

\usepackage[utf8]{inputenc}
\usepackage[french]{babel}
\usepackage[T1]{fontenc}
\usepackage[left=0.75in,top=0.6in,right=0.75in,bottom=0.6in]{geometry} % Document margins
\usepackage{enumitem}
\usepackage{hyperref}


\name{Quentin Bouniot} % Your name
\addressone{Paris, France 75014} % Your address
\addresstwo{quentin.bouniot@gmail.com} % Your phone number and email
\addressthree{\href{https://qbouniot.github.io}{Personal page} \\ \href{https://github.com/qbouniot}{Github} \\ \href{https://www.linkedin.com/in/quentin-bouniot/}{LinkedIn}} % Your secondary addess (optional)


\begin{document}

%----------------------------------------------------------------------------------------
%	EDUCATION SECTION
%----------------------------------------------------------------------------------------
\vspace{-15pt}

\begin{rSection}{Education}

\begin{rSubsection}{CEA-List, Université Paris-Saclay / Université Jean-Monnet, Saint-Étienne}{2019-2023}{PhD in Computer Science}{Paris, France}
    \item[] Thesis: "Towards Few-Annotation Learning in Computer Vision:\\ Application to Image Classification and Object Detection tasks",\\ Advisor: Prof. Amaury Habrard \\ \href{https://arxiv.org/abs/2311.04888}{Manuscript link}
    
\end{rSubsection}

% {\bf PhD in Computer Science} \hfill {\em 2019-Today} \\
% \emph{Few-Shot Learning: Application to Object Detection and Semantic Instance Segmentation} \\
% CEA List - Université Jean-Monnet-Saint-Étienne \\
% Advised by Prof. Amaury Habrard \\

\begin{rSubsection}{CentraleSupélec, Université Paris-Saclay}{2015-2019}{Engineering degree - M.Sc in Computer Science and Applied Mathematics}{Paris, France}
    \item[] 
    % Majoring in Robotics and AI
    \vspace{-15pt}
\end{rSubsection}

% {\bf CentraleSupélec, Paris, France} \hfill {\em 2015-2019} \\
% Master's Degree \\
% Majoring in Robotics and AI \smallskip \\

\begin{rSubsection}{Université de Lorraine}{2018-2019}{M.Sc in Computer Science and Vision - rank 1/18}{Nancy, France}
    \item[] 
    % Majoring in Machine Learning, Vision and Robotics, \\
            % with High Honors - rank 1/18
    \vspace{-15pt}
\end{rSubsection}

% {\bf Université de Lorraine, Nancy, France} \hfill {\em 2018-2019} \\
% Master's Degree \\
% Majoring in Machine Learning, Vision and Robotics \smallskip \\

\begin{rSubsection}{Higher School Preparatory Classes}{2013-2015}{Section Mathematics and Physics}{Bordeaux, France}
    \item[] 
    % Intensive preparatory course for national competitive exams to the \emph{grandes écoles}.
    \vspace{-15pt}
\end{rSubsection}
% {\bf Preparatory classes, Bordeaux, France} \hfill {\em 2013-2015} \\
% Section Mathematics and Physics \smallskip \\

\begin{rSubsection}{Baccalauréat}{2013}{Major in Mathematics}{Tahiti, French Polynesia}
    \item[]
    % \item[] with Highest Honors (\emph{Mention Très bien})
    \vspace{-15pt}
\end{rSubsection}
% {\bf Baccalauréat, Tahiti, French Polynesia} \hfill {\em 2013} \\
% Major in Mathematics \\
% High Honors \smallskip \\

\end{rSection}
\vspace{-18pt}


%----------------------------------------------------------------------------------------
%	WORK EXPERIENCE SECTION
%----------------------------------------------------------------------------------------

\begin{rSection}{Professional Experience}

\begin{rSubsection}{Telecom Paris, Institut Polytechnique de Paris}{February 2023 - Present}{Post-Doctoral Researcher}{Paris, France}
\item[] Working under the supervision of Florence d'Alché-Buc and Pavlo Mozharovskyi. \\
Uncertainty quantification and Explainability in Deep Learning. \\
Organizing the weekly team meetings.
    
\end{rSubsection}

\begin{rSubsection}{CEA-List, Université Paris-Saclay}{April - September 2019}{Research Intern}{Paris, France}

\item[] Working under the supervision of Romaric Audigier and Angélique Loesch. \\ Studying the impact of adversarial examples on person re-identification systems. \\
Improving the robustness of person re-identification systems using deep learning.
\end{rSubsection}

%------------------------------------------------

\begin{rSubsection}{SmartBuild Asia}{February - August 2018}{Intern - NLP, Summarization, Unsupervised matching}{Kuala Lumpur, Malaysia}
% \item[] Automatic gathering of online complex data and matching with existing information.
% \item[] Matching millions of unstructured sentences in both Malay and English to the construction project they were referring to.
\item[] 
% Summarization of web pages in a few sentences with high accuracy.
\vspace{-15pt}
\end{rSubsection}

%------------------------------------------------

\begin{rSubsection}{Orange France}{July 2017 - January 2018}{Intern - Conversational Agents, Software Engineering}{Paris, France}
\item[] 
% Development of tools for Conversational Agents.
% \item[] Architecture development for conversational agents that support collaborative learning.
\vspace{-15pt}
\end{rSubsection}
\vspace{-18pt}

\end{rSection}

% \pagebreak

\begin{rSection}{Teaching}

\begin{rSubsection}{Recent Developments in Responsible AI}{2023 - Now}{Institut Polytechnique de Paris}{Paris, France}
    \item[] Mini-course on \emph{Robust Machine Learning} as part of the \textit{M2 Data Science}.
    \begin{itemize}
        \item[$\cdot$] Adversarial Robustness ;
        \item[$\cdot$] Uncertainty Quantification.
    \end{itemize}
    \href{https://responsible-ai-datascience-ipparis.github.io/}{Ressources}
\end{rSubsection}

\begin{rSubsection}{Algorithms and complexity}{2020-2022}{Teaching Assistant at CentraleSupélec, Université Paris-Saclay}{Paris, France}
    \item[] First year Computer Science course for the main engineering track at CentraleSupélec. \\
    \href{https://wdi.centralesupelec.fr/1CC2000/}{Ressources}
\end{rSubsection}

\end{rSection}
\vspace{-15pt}


%----------------------------------------------------------------------------------------
%	PUBLICATIONS SECTION
%----------------------------------------------------------------------------------------

    \begin{rSection}{Selected Publications}

    \textsc{Preprints}

    \begin{itemize}[label=$\cdot$]
        \item \textbf{Quentin Bouniot}, Ievgen Redko, Anton Mallasto, Charlotte Laclau, Karol Arndt, Oliver Struckmeier, Markus Heinonen, Ville Kyrki, Samuel Kaski. "Understanding deep neural networks through the lens of their non-linearity." arXiv preprint 2310.11439 (2023). \href{https://arxiv.org/abs/2310.11439}{Paper link}
        \item \textbf{Quentin Bouniot}, Pavlo Mozharovskyi, Florence d'Alché-Buc. "Tailoring Mixup to Data using Kernel Warping functions." arXiv preprint 2311.01434 (2023). \href{https://arxiv.org/abs/2311.01434}{Paper link}

    \end{itemize}

    \textsc{International Conferences}

    \begin{itemize}[label=$\cdot$]

        \item \textbf{Quentin Bouniot}, Romaric Audigier, Angélique Loesch, Amaury Habrard. "Proposal-Contrastive Pretraining for Object Detection from Fewer Data." \emph{International Conference on Learning Representations (ICLR)}. 2023. \\ (\textbf{Oral - Notable top 25\%}) \href{https://openreview.net/forum?id=gm0VZ-h-hPy}{Paper link}

        \item \textbf{Quentin Bouniot}, Angélique Loesch, Romaric Audigier, Amaury Habrard. "Towards Few-Annotation Learning for Object Detection: Are Transformer-Based Models More Efficient?." \emph{Proceedings of the IEEE/CVF Winter Conference on Applications of Computer Vision (WACV).} 2023. \href{https://openaccess.thecvf.com/content/WACV2023/html/Bouniot_Towards_Few-Annotation_Learning_for_Object_Detection_Are_Transformer-Based_Models_More_WACV_2023_paper.html}{Paper link}

        \item \textbf{Quentin Bouniot}, Ievgen Redko, Romaric Audigier, Angélique Loesch, Amaury Habrard. "Improving Few-Shot Learning Through Multi-task Representation Learning Theory." \emph{Proceedings of the European Conference of Computer Vision (ECCV)}, 2022. \\ \href{https://arxiv.org/abs/2010.01992}{Paper link} \href{https://github.com/CEA-LIST/MetaMTReg}{Github link}

        \item \textbf{Quentin Bouniot}, Romaric Audigier, Angélique Loesch. "Optimal transport as a defense against adversarial attacks." \emph{2020 International Conference on Pattern Recognition (ICPR)}. IEEE, 2021. \href{https://arxiv.org/abs/2102.03156}{Paper link} \href{https://github.com/CEA-LIST/adv-sat}{Github link}

        \item \textbf{Bouniot Quentin}, Romaric Audigier, Angélique Loesch. "Vulnerability of person re-identification models to metric adversarial attacks." \emph{Proceedings of the IEEE/CVF Conference on Computer Vision and Pattern Recognition Workshops (CVPRW).} 2020. \\ (\textbf{DeepMind Travel Award.}) \href{https://openaccess.thecvf.com/content_CVPRW_2020/html/w47/Bouniot_Vulnerability_of_Person_Re-Identification_Models_to_Metric_Adversarial_Attacks_CVPRW_2020_paper.html}{Paper link} \href{https://github.com/CEA-LIST/adv-reid}{Github link}
    \end{itemize}

    \textsc{Patents}

    \begin{itemize}[label=$\cdot$]
        \item \textbf{Quentin Bouniot}, Romaric Audigier, Angélique Loesch. (2020) \emph{Méthode d'apprentissage d'un réseau de neurones pour le rendre robuste aux attaques par exemples contradictoires} (French Patent No. FR3116929A1). Institut national de la propriété industrielle (INPI).
        \href{https://data.inpi.fr/brevets/FR3116929}{Patent link}

        \item \textbf{Quentin Bouniot}, Romaric Audigier, Angélique Loesch. (2020) \emph{Learning method for a neural network for rendering it robust against attacks by contradictory examples} (European Patent No. EP4006786A1). European Patent Office (EPO). \href{https://worldwide.espacenet.com/publicationDetails/biblio?locale=en_EP&CC=EP&date=20220601&NR=4006786A1&ND=3&KC=A1&rnd=1699608498751&FT=D#}{Patent link}
    \end{itemize}

    \textsc{Communications}

    \begin{itemize}[label=$\cdot$]
    \item \textbf{Quentin Bouniot} \& Ievgen Redko "Understanding Few-Shot Multi-Task Representation Learning Theory", \emph{ICLR Blog Track}, 2022. \href{https://iclr-blog-track.github.io/2022/03/25/understanding_mtr_meta/}{Blog post link}
    \end{itemize}

    % \textsc{International Workshops}

    % \begin{itemize}[label=$\cdot$]
    %     \item \textbf{Bouniot, Quentin}, et al. "Putting Theory to Work: From Learning Bounds to Meta-Learning Algorithms." \emph{4th Workshop on Meta-Learning (MetaLearn) at NeurIPS 2020}.
    % \end{itemize}

    % \textsc{National Conferences}
    % \begin{itemize}[label=$\cdot$]
    %     \item \textbf{Bouniot, Quentin}, et al. "Vers une meilleure compréhension des méthodes de méta-apprentissage à travers la théorie de l'apprentissage de représentations multi-tâches." \emph{Conférence sur l'Apprentissage Automatique (CAp)}. 2021. \\ (\textbf{Oral Presentation})
    % \end{itemize}

\end{rSection}

\begin{rSection}{Academic Services}

\textsc{Member of Paris ELLIS Unit (\href{https://ellis.eu/units/paris}{Members})}
\begin{itemize}[label=$\cdot$]
    \item Evaluator for ELLIS Pre-screening PhD Program
\end{itemize}

\textsc{Peer Review}
\begin{itemize}[label=$\cdot$]
    \item Neural Information Processing Systems (NeurIPS), 2021-2023
    \item International Conference on Machine Learning (ICML), 2021-2024
    \item International Conference on Learning Representations (ICLR), 2022, 2024
    \item IEEE/CVF Winter Conference on Applications of Computer Vision (WACV), 2023
    \item NeurIPS Workshop on Meta-Learning (MetaLearn), 2020-2022
    \item ICML Workshop on Pre-training: Perspectives, Pitfalls, and Paths Forward, 2022
    \item International Conference on Automated Machine Learning (AutoML), 2022
    \item IEEE Transactions on Pattern Analysis and Machine Intelligence (TPAMI), \\ Special Issue on \emph{Learning with Fewer Labels in Computer Vision}, 2021
\end{itemize}

\textsc{Organizing Committee}
\begin{itemize}[label=$\cdot$]
    \item Workshop on \emph{Trustworthy and Frugal ML} with Jayneel Parekh, ELLIS Unconference 2023 in Paris (\href{https://ellisunconference2023.github.io/}{Link to the event})
    \item Tutorial on Uncertainty Quantification at WACV 2024: \emph{The Nuts and Bolts of Deep Uncertainty Quantification}, with Gianni Franchi, Olivier Laurent, and Andrei Bursuc. (\href{https://uqtutorial.github.io/}{Link to the event})
\end{itemize}

\textsc{Open-Source}
\begin{itemize}[label=$\cdot$]
    \item Developer for \href{https://github.com/ENSTA-U2IS/torch-uncertainty}{torch-uncertainty}: Comprehensive PyTorch Library for deep learning uncertainty quantification techniques. 
\end{itemize}

    
\end{rSection}
    
    %----------------------------------------------------------------------------------------
    %	TALKS SECTION
    %----------------------------------------------------------------------------------------
    
    \begin{rSection}{Oral Presentations}

    \begin{rSubsection}{Ecole Polytechnique - CMAP Seminar}{2023}{}{}
        \item[] On Few-Annotation Learning and Non-Linearity in Deep Neural Networks
    \end{rSubsection}

    \begin{rSubsection}{ELLIS Unconference - Plenary talk}{2023}{}{}
        \item[] Towards Few-Annotation Learning in Computer Vision: Application to Image Classification \\ and Object Detection tasks
    \end{rSubsection}

    \begin{rSubsection}{DSAIDIS Chair - Workshop Frugality in Machine Learning}{2023}{}{}
        \item[] Towards better understanding meta-learning methods through multi-task representation learning theory.
    \end{rSubsection}

    \begin{rSubsection}{CAp - French Machine Learning Conference}{2023}{}{}
            \item[] Proposal-Contrastive Pretraining for Object Detection from Fewer Data.
    \end{rSubsection}
    
    \begin{rSubsection}{CAp - French Machine Learning Conference}{2021}{}{}
            \item[] Towards better understanding meta-learning methods through multi-task representation learning theory.
    \end{rSubsection}

    \begin{rSubsection}{CEA-List, Université Paris-Saclay}{2022}{}{}
        \item[] Factory-AI for Deep Learning Purposes. 
        % \\ Tutorial for several CEA-List laboratories on the use of the internal HPC cluster based on Slurm for deep learning experiments.
    \end{rSubsection}

    \begin{rSubsection}{GdR ISIS - Towards pragmatic learning in a context of limited labeled visual data}{2021}{}{}
    \item[] Improving Few-Shot Learning through Multi-task Representation Learning Theory.
    \end{rSubsection}

    \begin{rSubsection}{NeurIPS - Workshop on Meta-Learning (MetaLearn)}{2020}{}{}
        \item[] Putting Theory to Work: From Learning Bounds to Meta-Learning Algorithms.
\end{rSubsection}

    \begin{rSubsection}{DataIA - Workshop "Safety \& AI"}{2020}{}{}
        \item[] Vulnerability of person re-identification models to metric adversarial attacks.
    \end{rSubsection}
    
    \end{rSection}

    \newpage

%----------------------------------------------------------------------------------------
%	TECHNICAL STRENGTHS SECTION
%----------------------------------------------------------------------------------------

\begin{rSection}{Technical Skills}

    \begin{tabular}{ @{} >{\bfseries}l @{\hspace{6ex}} l }
    Computer Languages & Python, C++, Matlab, Java, Javascript \\
    Machine Learning & Pytorch, Keras, Tensorflow, Scikit-Learn \\
    HPC & Slurm \\
    Systems & Unix, Windows, SQL, Git
    \end{tabular}
    
    \end{rSection}
    
    %----------------------------------------------------------------------------------------
    %	LANGUAGES SECTION
    %----------------------------------------------------------------------------------------
    
    \begin{rSection}{Languages}
    
    \begin{tabular}{ @{} >{\bfseries}l @{\hspace{6ex}} l }
    French & Native \\
    English & Excellent - C1 \\
    Japanese & Studying - A2
    \end{tabular}
    
    \end{rSection}

%----------------------------------------------------------------------------------------
%	VOLUNTARY SERVICE SECTION
%----------------------------------------------------------------------------------------

% \begin{rSection}{Voluntary Services}

% \begin{rSubsection}{Supélec Rézo}{2015 - 2017}{Executive Board member}{Paris, France}
% \item[] Management and maintenance of the IT and network infrastructure of the CentraleSupélec campus (about 1000 users).
% \item[] Administrator of several servers.
% % \item[] \vfill
% \end{rSubsection}

% \begin{rSubsection}{CentraleSupélec Gaming club}{2015 - 2017}{President}{Paris, France}
% \item[] Manage, organize and promote video-games events on the CentraleSupélec campus.
% \end{rSubsection}

% \begin{rSubsection}{Student Gaming Network}{2015 - 2017}{Member}{Paris, France}
% \item[] Federation of French students' associations of eSport.
% \item[] Organize every year the \emph{Student Gaming League}, an eSport tournament for students in France.
% \end{rSubsection}
% \end{rSection}


%----------------------------------------------------------------------------------------

\end{document}
